\chapter{第一章题目}

本章的主要内容与学校提供的Word模板中内容一致,图片与表格均采用原始设定大小,%
主要是为了说明格式的统一。%
但是,\LaTeX{}的一些禁则,专业排版的能力,对公式及文献的处理都是得天独厚的,%
我们不必刻意去追求与Word的完美匹配。而且你将会发现,用\LaTeX{}书写论文的美! %

\section{(1.1 题目)}
正文内容

\subsection{(1.1.1 题目)}
正文内容

正文内容

\begin{figure}[htp]
\centering
\includegraphics{picmain}
\caption{图 1.1 名称}
\end{figure}

\subsubsection{(1.1.1.1 题目)}
正文内容

正文内容

正文内容

\subsubsection{(1.1.1.2 题目)}
正文内容

正文内容

正文内容

\subsection{(1.1.2 题目)}
正文内容

正文内容

\begin{figure}[htp]
\centering
\includegraphics{picmain}
\caption{图 1.2 名称}
\end{figure}

\section{(1.2 题目)}
正文内容

正文内容

\begin{table}[htp]
\centering
\caption{表 1.2 名称}
\begin{tabular}{|c|c|c|c|c|}
\hline
\makebox[2.07cm][0pt]{} & \makebox[2.07cm][0pt]{} & \makebox[2.07cm][0pt]{} & \makebox[2.07cm][0pt]{} & \makebox[2.07cm][0pt]{} \\
\hline
 & & & & \\
\hline
 & & & & \\
\hline
\end{tabular}
\end{table}

正文内容

正文内容

正文内容

正文内容

\section{(1.3 题目)}
正文内容

正文内容

正文内容

正文内容

正文内容

正文内容

\subsection{(1.3.1 题目)}
正文内容

\begin{figure}[htp]
\centering

\resizebox{0.5\textwidth}{!}{%校徽
\begin{tikzpicture}[align=center]
%\draw[ultra thick] circle[radius=6cm] circle[radius=7cm] ;

\draw[line width=2mm,yellow!70!red,fill=blue!50!black] circle[radius=5.4cm] ;
 \draw[line width=1pt,white,fill=white] (0.5cm,-3cm) .. controls (3,-2.5)and(5,-3.5)  .. (4cm,-4cm)-- (3.5cm ,-4.5cm) .. controls (2.3,-3.3)  ..cycle;


  \draw[line width=1pt,white,fill=white] (0.55cm,-2.8cm) .. controls (3,-1.2)and(5,-2)  .. (6cm,-1.6cm)-- (6cm ,-3.5cm) .. controls (3cm,-2.5cm)  ..cycle;

\draw[line width=1pt,white,fill=white] (0.6cm,-2.6cm) .. controls (2cm,-1.2cm)and(3cm,0cm)  .. (6cm,0.5cm)-- (6cm ,-1.2cm) .. controls (5cm,-0.7cm)and (1cm,-2cm) ..cycle;

 \draw[line width=2pt,white,fill=white] (1.6cm,1.2cm)-- (0cm,-3.2cm)--(-2.2cm,-4.2cm)--(-0.6cm,0.2cm)-- cycle;

 \draw[line width=1pt,white,fill=white] (-0.8cm,0.2cm) .. controls (-1cm,0cm)and(-3cm,0cm)  .. (-5cm,5cm)-- (-6cm ,2.5cm) .. controls (-4cm,1cm)and (-2cm,-0.6cm) ..cycle;

  \draw[line width=1pt,white,fill=white] (-0.85cm,0cm) .. controls (-1cm,-0.4cm)and(-3cm,-0.2cm)  .. (-6cm,2cm)-- (-6cm ,0cm) .. controls (-4cm,-0.4cm)and (-2cm,-1cm) ..cycle;
  \draw[line width=1pt,white,fill=white] (-0.9cm,-0.2cm) .. controls (-2cm,-1cm)and(-3cm,-0.7cm)  .. (-6cm,-0.7cm)-- (-6cm ,-2cm) .. controls (-4cm,-2cm)and (-2cm,-1.5cm) ..cycle;
 \draw[line width=2mm,yellow!70!red,fill=none] circle[radius=5.4cm] ;
\draw[line width=2.5mm,yellow!70!red] circle[radius=7.4cm];
\path
[rotate=0,postaction={decoration={text along path,text format delimiters={|}{|}, text={ |\erhao \color{blue!50!black}\ehei| INFORMATION ENGINEERING UNIVERSITY },
text align=fit to path}, decorate}]
(-6.3cm,0) arc (180:0:6.3cm and 6.3cm) ;

\path
[rotate=40,postaction={decoration={text along path,text format delimiters={|}{|}, text={|\xiaochu \color{red}\hei |  信息工程大学 },
text align=fit to path}, decorate}]
(-6.8cm,0) arc (-180:-90:6.8cm and 6.8cm) ;

\node [star, star point height=1.5cm, minimum size=5cm, draw=yellow!70!red,fill=red,line width=1.2mm]
at (0,3.6cm) {};

\draw[line width=1pt,yellow!70!red,fill=yellow!70!red] (-0.4cm,4.2cm) -- (0.38cm,4.2cm)--(0.4cm,4.22cm)--(0.42cm,4.18cm)--(0.4cm,4.15cm)..controls(0.4cm,3.9cm)..(0.5cm,3.5cm)--(0.4cm,3.4cm)..controls(0.3cm,3.9cm)..(0.3cm,4.1cm)--(-0.4cm,4.1cm)--cycle;

\draw[line width=1pt,yellow!70!red,fill=yellow!70!red](-0.38cm,4cm)--(-0.28,3.98cm)..controls(-0.28cm,3.8cm)and(-0.33cm,3.6cm)..(-0.5cm,3.4cm)..controls(-0.43cm,3.6cm)..cycle;
\draw[line width=1pt,yellow!70!red,fill=yellow!70!red](-0.5cm,3.2cm)--(0.4cm,3.2cm)--(0.45cm,3.25cm)--(0.5cm,3.15cm)--(0.48cm,3.08cm)--(0.45cm,3.1cm)--(-0.5cm,3.1cm)--cycle;
\end{tikzpicture} }
\caption{图 1.3 名称}
\end{figure}

\subsection{(1.3.2 题目)}
正文内容

正文内容

\begin{table}[htp]
\centering
\caption{表 1.2 名称}
\begin{tabular}{|c|c|c|c|c|}
\hline
\makebox[2.07cm][0pt]{} & \makebox[2.07cm][0pt]{} & \makebox[2.07cm][0pt]{} & \makebox[2.07cm][0pt]{} & \makebox[2.07cm][0pt]{} \\
\hline
 & & & & \\
\hline
 & & & & \\
\hline
\end{tabular}
\end{table}

